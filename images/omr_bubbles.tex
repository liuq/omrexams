\documentclass[10pt]{standalone}
\usepackage{graphicx}
\usepackage[ocr-b]{ocr}
\usepackage{tikz}
\usetikzlibrary{shapes.misc}
\usetikzlibrary{calc}
\usetikzlibrary{positioning}
\usetikzlibrary{decorations.markings}


\newcounter{omranswers}
\newcommand{\omrchoices}[1]{%
\begin{tikzpicture}[x=14pt, y=14pt]
\foreach \position in {1, ..., #1} {%
  \setcounter{omranswers}{\position}
%\node [draw, fill=black, circle, inner sep=0pt, minimum size=14pt, text=white, font=\small] at (0, 0) {\thequestion};
  \node [circle, inner sep=0pt, minimum size=14pt, color=gray] at ($({1.25 * \position}, 1.0)$) {\small\ocr{\Alph{omranswers}}};
  \node [draw=black!50!white, line width=2pt, circle, inner sep=0pt, minimum size=14pt] at ($({1.25 * \position}, 0)$) {};
};
\end{tikzpicture}
}

\begin{document}
\begin{sf}
\bgroup
%\def\arraystretch{0.1}
\begin{tabular}{cccccc}
%\includegraphics[width=1.5cm]{omr_ok_x.png} & 
%\includegraphics[width=1.5cm]{omr_no_x1.png} & 
%\includegraphics[width=1.5cm]{omr_no_x2.png} &
%\includegraphics[width=1.5cm]{omr_no_x3.png} &
%\includegraphics[width=1.5cm]{omr_no_x4.png} \\
\omrchoices{3} & \omrchoices{3} & \omrchoices{3} & \omrchoices{3} & \omrchoices{3} & \omrchoices{3} \\
%Ok & No & No & No & No & No \\
\end{tabular}
\egroup
\end{sf}
\end{document}
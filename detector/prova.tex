\documentclass{omrexam}
\usepackage{polyglossia}
\setdefaultlanguage{italian}

\examname{Fondamenti di Informatica}
\student{34601}{Luca Di Gaspero}
\date{20/07/2019}
\solution{edhcaedhc}

\begin{document}
\begin{questions}
\question
Cos'è un attributo? \omrchoices{3}
\begin{choices}

\choice 
un parametro usato all'interno dei tag per specificare informazioni aggiuntive

\choice 
una proprietà di un'entità

\choice 
il colore di sfondo di una pagina $\phi = \sqrt{2}$

\end{choices}
\question
Cos'è un'ancora? \omrchoices{3}
\begin{choices}

\choice 
un tag HTML che specifica un collegamento

\choice 
il testo associato al riferimento evidenziato nel documento

\choice 
una ripetizione di uno o più tag

\end{choices}
\question
Di cosa si occupa il World-Wide Web Consortium (W3C) \omrchoices{3}
\begin{choices}

\choice 
di definire standard mondiali riguardanti il Web

\choice 
l'assegnazione dei nomi di dominio

\choice 
gestisce gli indirizzi IP statici

\end{choices}
\question
Un cookie... \omrchoices{3}
\begin{choices}

\choice 
è un informazione utilizzata da un server web per registrare informazioni su un computer client

\choice 
viene impostato dal server web e reinviato dal client ad ogni successiva richiesta

\choice 
è parte fondamentale del protocollo HTTP

\end{choices}
\begin{tabular}{l c r}
Tables & Are & Cool \\
\hline
col 3 is & right-aligned & \$1600 \\
col 2 is & centered & \$12 \\
zebra stripes & are neat & \$1 \\
\end{tabular}

\end{questions}
\end{document}
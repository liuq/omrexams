\documentclass{omrexam}
\usepackage{polyglossia}
\setdefaultlanguage{italian}

\examname{Fondamenti di Informatica – Relazioni Pubbliche}
\student{34601}{Luca Di Gaspero}
\date{21/02/2019}
\solution{A,AB,A}

\begin{document}

\textbf{Regole per il compito} Non si copia, si deve stare zitti, non si possono usare i cellulari.

\begin{questions}
\question
L'unità Aritmetico/Logica (ALU) \omrchoices{3}
\begin{choices}

\choice 
è un sottosistema del computer che esegue le operazioni specificate in un'istruzione macchina

\choice 
è in grado di fare calcoli complessi

\choice 
fa parte della CPU

\end{choices}
\question
Cos'è un'ancora? \omrchoices{3}
\begin{choices}

\choice 
un tag HTML che specifica un collegamento

\choice 
il testo associato al riferimento evidenziato nel documento

\choice 
una ripetizione di uno o più tag

\end{choices}
\question
L'unità di output \omrchoices{3}
\begin{choices}

\choice 
è un sottosistema del computer che trasferisce informazione dalla memoria al mondo fisico

\choice 
gestisce le periferiche di output

\choice 
usa un bus

\end{choices}

\end{questions}
\end{document}
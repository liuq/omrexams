\documentclass{omrexam}

\usepackage{lipsum}
\usepackage{polyglossia}
\setdefaultlanguage{italian}

\examname{Fondamenti di Informatica per Relazioni Pubbliche}
%\student{20885}{Andrea Fusiello}
\student{34601}{Luca Di Gaspero}
\date{17/02/2019}
%\solution{A, BC, A, D}
\solution{edidaedk}

\begin{document}  

\begin{questions}
\question[10]
Cos'è un'ancora? \dotfill \omrchoices{3}

\begin{choices}
\choice 
un tag HTML che specifica un collegamento

\choice 
il testo associato al riferimento evidenziato nel documento

\choice 
una ripetizione di uno o più tag

\end{choices} 
 
\question[10]
Cos'è un'ancora? Ma vorrei che fosse un testo più generoso per verificare cosa succede quando si va a capo, vediamo, non lo so proprio \dotfill \omrchoices{5}


\begin{choices}
\choice 
un tag HTML che specifica un collegamento

\choice 
il testo associato al riferimento evidenziato nel documento

\choice 
una ripetizione di uno o più tag

\end{choices}

\question[10]
Cos'è un'ancora? \dotfill \omrchoices{4}

\begin{choices}
\choice 
un tag HTML che specifica un collegamento

\choice 
il testo associato al riferimento evidenziato nel documento

\choice 
una ripetizione di uno o più tag

\end{choices}

\lipsum

\question[10]
Cos'è un'ancora? \dotfill \omrchoices{4}

\begin{choices}
\choice 
un tag HTML che specifica un collegamento

\choice 
il testo associato al riferimento evidenziato nel documento

\choice 
una ripetizione di uno o più tag

\end{choices}

\end{questions}

\end{document}